\documentclass{article}
\usepackage{graphicx} % Required for inserting images
\usepackage{hyperref}
\usepackage{listings}
\usepackage{color}

\title{1807D - Odd Queries} 

% Author must be your full name
\author{Ferrel Destatiananda Edwardo} 

% Date is when you create this report
\date{23 April 2024}

\begin{document}

\maketitle

% There are 4 Sections, Problem, Objective, Solution, Code

% Problem section contains hyperlink to the problem
\section{Problem}

Problem Description : \href{https://codeforces.com/problemset/problem/1807/D}{https://codeforces.com/problemset/problem/1807/D}

% Objective section contains what is the problem's objective
\section{Objective}

The objective of this problem is to determine whether changing all elements in specified ranges of an array to a given value will result in the sum of the entire array being odd.

% Solution section contains how you approch the problem and your solution
\section{Solution}
The solution is using segment tree.
1. Initialize a segment tree with the given array.
2. When we input the array, we compute the sum of all elements in the array.
3. For every query, we first subtract the sum of elements within the query range from the total initial sum of the array and we the replacements.
4. Combining these steps, we obtain the sum of the entire array after applying the replacements, represented as (sum - query + replacements).
4. If the sum of query is odd, we output "YES". Otherwise, we output "NO".

% Code section contains your solution code

\newpage
\section{Code}

\lstset{language=C++,
        basicstyle=\ttfamily,
        keywordstyle=\color{blue}\ttfamily,
        stringstyle=\color{red}\ttfamily,
        commentstyle=\color{green}\ttfamily,
        morecomment=[l][\color{magenta}]{\#}
}
\begin{lstlisting}
#include<bits/stdc++.h>

using namespace std;

struct segmenttree {
	int n;
	vector<int> st;

	void init(int _n) {
		this->n = _n;
		st.resize(4 * n, 0);
	}

	void build(int start, int ending, int node, vector<int> &v) {
		// leaf node base case
		if (start == ending) {
			st[node] = v[start];
			return;
		}

		int mid = (start + ending) / 2;

		// left subtree is (start,mid)
		build(start, mid, 2 * node + 1, v);

		// right subtree is (mid+1,ending)
		build(mid + 1, ending, 2 * node + 2, v);

		st[node] = st[node * 2 + 1] + st[node * 2 + 2];
	}

	int query(int start, int ending, int l, int r, int node) {
		// non overlapping case
		if (start > r || ending < l) {
			return 0;
		}

		// complete overlap
		if (start >= l && ending <= r) {
			return st[node];
		}

		// partial case
		int mid = (start + ending) / 2;

		int q1 = query(start, mid, l, r, 2 * node + 1);
		int q2 = query(mid + 1, ending, l, r, 2 * node + 2);

		return q1 + q2;
	}


	void build(vector<int> &v) {
		build(0, n - 1, 0, v);
	}

	int query(int l, int r) {
		return query(0, n - 1, l, r, 0);
	}

};

int main() {
    int t;
    cin >> t;
    while (t--) {
        int n, q;
        cin >> n >> q;
        vector<int> v(n);
        int sum = 0;
        for (int i = 0; i < n; i++) {
            cin >> v[i];
            sum += v[i];
        }

        segmenttree tree;
        tree.init(n);
        tree.build(v);

        for (int i = 0; i < q; i++) {
            int l, r, k;
            cin >> l >> r >> k;
            if ((sum - tree.query(l - 1, r - 1) + k * (r - l + 1)) % 2 != 0) {
                cout << "YES" << endl;
            } else {
                cout << "NO" << endl;
            }
        }
    }
    return 0;
}

\end{lstlisting}

\end{document}