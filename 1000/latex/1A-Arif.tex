\documentclass{article}
\usepackage{graphicx} % Required for inserting images
\usepackage{hyperref}
\usepackage{listings}
\usepackage{color}

% Example : https://codeforces.com/problemset/problem/266/A 
% So the title would be 266A - Stones on the Table
\title{1A -  Theatre Square} 

% Author must be your full name
\author{Muhammad Arif Rifki} 

% Date is when you create this report
\date{17 April 2024}

\begin{document}

\maketitle

% There are 4 Sections, Problem, Objective, Solution, Code

% Problem section contains hyperlink to the problem
\section{Problem}

Problem Description : \href{https://codeforces.com/problemset/problem/1/A}{https://codeforces.com/problemset/problem/1/A}

% Objective section contains what is the problem's objective
\section{Objective}
Theatre Square in the capital city of Berland has a rectangular shape with the size $nxm$ meters. On the occasion of the city's anniversary, a decision was taken to pave the Square with square granite flagstones. Each flagstone is of the size $axa$.

What is the least number of flagstones needed to pave the Square? It's allowed to cover the surface larger than the Theatre Square, but the Square has to be covered. It's not allowed to break the flagstones. The sides of flagstones should be parallel to the sides of the Square.

\\\textbf{Input}
The input contains three positive integer numbers in the first line: $n$,$m$ and $a$ ($1 \leq n, m, a \leq 10^9$).

\\\textbf{Output}
Write the needed number of flagstones.
% Solution section contains how you approch the problem and your solution
\section{Solution}

to cover all the shape, we just have to, count how many $m$ is needed to cover the length and width.

% Code section contains your solution code

\newpage
\section{Code}

\lstset{language=C++,
        basicstyle=\ttfamily,
        keywordstyle=\color{blue}\ttfamily,
        stringstyle=\color{red}\ttfamily,
        commentstyle=\color{green}\ttfamily,
        morecomment=[l][\color{magenta}]{\#}
}
\begin{lstlisting}
#include <bits/stdc++.h>
using namespace std;
int main()
{
    long long n, m, a;
    cin >> n >> m >> a;
    long long temp1 = (n + a - 1) / a;
    long long temp2 = (m + a - 1) / a;
    cout << temp1 * temp2<<endl;
    return 0;
}

\end{lstlisting}

\end{document}