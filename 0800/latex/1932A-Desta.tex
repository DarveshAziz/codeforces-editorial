\documentclass{article}
\usepackage{graphicx} % Required for inserting images
\usepackage{hyperref}
\usepackage{listings}
\usepackage{color}

\title{1932A - Thorns and Coins} 

% Author must be your full name
\author{Ferrel Destatiananda Edwardo} 

% Date is when you create this report
\date{22 April 2024}

\begin{document}

\maketitle

% There are 4 Sections, Problem, Objective, Solution, Code

% Problem section contains hyperlink to the problem
\section{Problem}

Problem Description : \href{https://codeforces.com/problemset/problem/1932/A}{https://codeforces.com/problemset/problem/1932/A}

% Objective section contains what is the problem's objective
\section{Objective}

The objective is to find the maximum number of coins that can be collected while traversing a path with thorns and coins. The path can be traversed by moving one or two cells at a time, but thorny cells cannot be stepped on.

% Solution section contains how you approch the problem and your solution
\section{Solution}

Iterate through the path. For each cell: 1. If the cell contains a coin ('@'), increment the count of collected coins. 2. If the cell contains thorns ('*'), check the next cell. If it also contains thorns, break the loop since it's not possible to proceed further without stepping on thorns.

% Code section contains your solution code

\newpage
\section{Code}

\lstset{language=C++,
        basicstyle=\ttfamily,
        keywordstyle=\color{blue}\ttfamily,
        stringstyle=\color{red}\ttfamily,
        commentstyle=\color{green}\ttfamily,
        morecomment=[l][\color{magenta}]{\#}
}
\begin{lstlisting}
#include <bits/stdc++.h>
#define fastio ios_base::sync_with_stdio(false); cin.tie(NULL);
using namespace std;

int solve(char path[], int n) {
    int ans = 0;
    for(int i = 0; i < n; i++){
        if(path[i]=='@') ans++;
        else if (path[i]=='*'){
            if(path[i+1]=='*') break;
        }
    }
    return ans;
}

int main(){
    fastio;
    int t;
    cin >> t;


    while(t--){
        int n;
        cin >> n;
        char path[n];
        for(int i = 0; i < n; i++){
            cin >> path[i];
        }
        cout << solve(path, n) << endl;
    }

    return 0;
}

\end{lstlisting}

\end{document}
