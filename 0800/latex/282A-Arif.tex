\documentclass{article}
\usepackage{graphicx} % Required for inserting images
\usepackage{hyperref}
\usepackage{listings}
\usepackage{color}

% Example : https://codeforces.com/problemset/problem/266/A 
% So the title would be 266A - Stones on the Table
\title{282A - Bit++} 

% Author must be your full name
\author{Muhammad Arif Rifki} 

% Date is when you create this report
\date{17 April 2024}

\begin{document}

\maketitle

% There are 4 Sections, Problem, Objective, Solution, Code

% Problem section contains hyperlink to the problem
\section{Problem}

Problem Description : \href{https://codeforces.com/problemset/problem/282/A}{https://codeforces.com/problemset/problem/282/A}

% Objective section contains what is the problem's objective
\section{Objective}
The classic programming language of Bitland is Bit++. This language is so peculiar and complicated.

The language is that peculiar as it has exactly one variable, called x. Also, there are two operations:

\begin{itemize}
    \item Operation ++ increases the value of variable x by 1.
    \item Operation -- decreases the value of variable x by 1.
\end{itemize}

A statement in language Bit++ is a sequence, consisting of exactly one operation and one variable x. The statement is written without spaces, that is, it can only contain characters "+", "-", "X". Executing a statement means applying the operation it contains.

A programme in Bit++ is a sequence of statements, each of them needs to be executed. Executing a programme means executing all the statements it contains.

You're given a programme in language Bit++. The initial value of x is 0. Execute the programme and find its final value (the value of the variable when this programme is executed).

\\\textbf{Input}
The first line contains a single integer n ($1 \leq n \leq 150$
) — the number of statements in the program.

Next n lines contain a statement each. Each statement contains exactly one operation (++ or --) and exactly one variable x (denoted as letter «X»). Thus, there are no empty statements. The operation and the variable can be written in any order.

\\\textbf{Output}
Print a single integer — the final value of x.

% Solution section contains how you approch the problem and your solution
\section{Solution}

Since the initial value is 0 (we call it $a$), and for each statement if the statement contains a '+' sign, then add 1 to a, and if it contains '-', subtract a by 1.

% Code section contains your solution code

\newpage
\section{Code}

\lstset{language=C++,
        basicstyle=\ttfamily,
        keywordstyle=\color{blue}\ttfamily,
        stringstyle=\color{red}\ttfamily,
        commentstyle=\color{green}\ttfamily,
        morecomment=[l][\color{magenta}]{\#}
}
\begin{lstlisting}
#include <bits/stdc++.h>
using namespace std;
int main(){
    int x, cnt = 0;
    cin >> x;
    string s;
    for (int i = 0; i < x; i++){
        cin >> s;
        if (s.find("+") != string::npos)
            cnt++;
        else
            cnt--;
    }
    cout << cnt;
    return 0;
}
\end{lstlisting}

\end{document}