\documentclass{article}
\usepackage{graphicx} % Required for inserting images
\usepackage{hyperref}
\usepackage{listings}
\usepackage{color}

% Example : https://codeforces.com/problemset/problem/266/A 
% So the title would be 266A - Stones on the Table
\title{266A - Stones on the Table} 

% Author must be your full name
\author{Devan Ferrel} 

% Date is when you create this report
\date{8 April 2024}

\begin{document}

\maketitle

% There are 4 Sections, Problem, Objective, Solution, Code

% Problem section contains hyperlink to the problem
\section{Problem}

Problem Description : \href{https://codeforces.com/problemset/problem/266/A}{https://codeforces.com/problemset/problem/266/A}

% Objective section contains what is the problem's objective
\section{Objective}

The objective of this problem is literally simple, we just have to count the minimum number of stones we have to remove from the string 
so that any neighboring stones had different colors.

Example : 

$s$ = RRRG

In the string $s$, $s_1$ has same color as $s_0$, so we remove it from the string and the string will look like

$s$ = RRG

This time, the new $s_1$ also has same color as $s_0$, so we remove it from the string and the result would be 

$s$ = RG

Since theres is no any neighboring characters in $s_0$ that has same color as $s_0$, we dont have to remove any character from string $s$ anymore, so the answer is 2

% Solution section contains how you approch the problem and your solution
\section{Solution}

We can use sliding window technique for this problem.
Firstly, we create a variable $l$, initially it points to index 0 and then we start the loop from 1 to $n$. We compare $s_l$ with $s_i$ and if they are same, then we increment the counter $c$, else we move the pointer $l$ to $i$. 

% Code section contains your solution code

\newpage
\section{Code}

\lstset{language=C++,
        basicstyle=\ttfamily,
        keywordstyle=\color{blue}\ttfamily,
        stringstyle=\color{red}\ttfamily,
        commentstyle=\color{green}\ttfamily,
        morecomment=[l][\color{magenta}]{\#}
}
\begin{lstlisting}

#include <bits/stdc++.h>

int main(){
    fastio
    int n,l=0,c=0;
    string s;
    cin >> n >> s;

    for(int i = 1; i < n; i++) {
        if(s[i] == s[l]) {
            c++;
        } else {
            l = i;
        }
    }

    cout << c;

    return 0;
}

\end{lstlisting}

\end{document}
